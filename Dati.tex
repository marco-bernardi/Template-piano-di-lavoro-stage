%----------------------------------------------------------------------------------------
%   USEFUL COMMANDS
%----------------------------------------------------------------------------------------

\newcommand{\dipartimento}{Dipartimento di Matematica ``Tullio Levi-Civita''}

%----------------------------------------------------------------------------------------
% 	USER DATA
%----------------------------------------------------------------------------------------

% Data di approvazione del piano da parte del tutor interno; nel formato GG Mese AAAA
% compilare inserendo al posto di GG 2 cifre per il giorno, e al posto di 
% AAAA 4 cifre per l'anno
\newcommand{\dataApprovazione}{28/03/2023}

% Dati dello Studente
\newcommand{\nomeStudente}{Marco}
\newcommand{\cognomeStudente}{Bernardi}
\newcommand{\matricolaStudente}{2018528}
\newcommand{\emailStudente}{marco.bernardi.11@studenti.unipd.it}
\newcommand{\telStudente}{+ 39 338 44 71 992}

% Dati del Tutor Aziendale
\newcommand{\nomeTutorAziendale}{Federico}
\newcommand{\cognomeTutorAziendale}{Milan}
\newcommand{\emailTutorAziendale}{milan.federico@breton.it}
\newcommand{\telTutorAziendale}{+ 39 334 63 68 798}
\newcommand{\ruoloTutorAziendale}{Software Product Manager}

% Dati dell'Azienda
\newcommand{\ragioneSocAzienda}{Breton S.p.A}
\newcommand{\indirizzoAzienda}{Via Garibaldi 27, Castello di Godego (TV)}
\newcommand{\sitoAzienda}{https://breton.it/}
\newcommand{\emailAzienda}{info@breton.it}
\newcommand{\partitaIVAAzienda}{P.IVA 01880270267}

% Dati del Tutor Interno (Docente)
\newcommand{\titoloTutorInterno}{Prof.}
\newcommand{\nomeTutorInterno}{Lamberto}
\newcommand{\cognomeTutorInterno}{Ballan}

\newcommand{\prospettoSettimanale}{
     % Personalizzare indicando in lista, i vari task settimana per settimana
     % sostituire a XX il totale ore della settimana
    \begin{itemize}
        \item \textbf{Prima Settimana (40 ore) - Kick off}
        \begin{itemize}
            \item Inserimento nella Business unit Digital Hub di Breton SpA; 
	    \item Definizione obietti e requisiti del progetto;
            \item Verifica credenziali e presentazione della politica aziendale per la sicurezza e modalità di utilizzo degli strumenti aziendali assegnati;
            \item Definizione roadmap di progetto e kick off;
            \item Formazione on the job su tecnologie e strumenti utili per il progetto;
        \end{itemize}
        \item \textbf{Seconda Settimana - Ricerca e documentazione (40 ore)} 
        \begin{itemize}
            \item Ricerca e documentazione su GANs ed affini;
            \item Ricerca di esempi e casi d'uso su GANs e affini;
        \end{itemize}
        \item \textbf{Terza Settimana - Ricerca e documentazione (40 ore)} 
        \begin{itemize}
            \item Ricerca e documentazione su GANs ed affini;
            \item Ricerca di esempi e casi d'uso su GANs ed affini;
            \item Check degli obiettivi definiti su road map;
        \end{itemize}
        \item \textbf{Quarta Settimana - Preparazione ambienti di analisi e sviluppo (40 ore)} 
        \begin{itemize}
            \item Scelta dell'ambiente di analisi e di sviluppo e relativa preparazione;
            \item Test dell'ambiente e presa confidenza;
        \end{itemize}
        \item \textbf{Quinta Settimana - Scelta set di dati (40 ore)} 
        \begin{itemize}
            \item Scelta del set di dati da utilizzare;
            \item Pulizia e preparazione del dato partendo dal set di dati;
            \item Selezione del miglior set di dati per training e test;
        \end{itemize}
        \item \textbf{Sesta Settimana - Sviluppo (40 ore)} 
        \begin{itemize}
            \item Sviluppo e test in cicli interattivi del POC;
        \end{itemize}
        \item \textbf{Settima Settimana - Sviluppo (40 ore)} 
        \begin{itemize}
            \item Sviluppo e test in cicli interattivi del POC;
            \item Raccolta e analisi dei risultati;
        \end{itemize}
        \item \textbf{Ottava Settimana - Conclusione (40 ore)} 
        \begin{itemize}
            \item Verifica dei risultati finali;
            \item Stesura della documentazione di progetto;
        \end{itemize}
    \end{itemize}
}

% Indicare il totale complessivo (deve essere compreso tra le 300 e le 320 ore)
\newcommand{\totaleOre}{320}

\newcommand{\obiettiviObbligatori}{
	 \item \underline{\textit{O01}}: Raggiungere un pensiero analitico e sistemico multidisciplinare grazie alla scomposizione del problema e sviluppo di una soluzione modulare;
	 \item \underline{\textit{O02}}: Autonomia nella gestione progettuale con capacità di sintesi del problema e propositività verso soluzioni;
	 \item \underline{\textit{O03}}: Qualità nella produzione di artefatti tecnologici e/o software e relativa documentazione;
	 
}

\newcommand{\obiettiviDesiderabili}{
	 \item \underline{\textit{D01}}: Sviluppo di un POC realtivo alla generazione di immagini attraverso l'uso di GANs o tecnologie alternative;
	 \item \underline{\textit{D02}}: Sviluppo di un POC relativo all'aumento risoluzione di una immagine;
	 \item \underline{\textit{D02}}: Validazione degli artefatti prodotti;
}

\newcommand{\obiettiviFacoltativi}{
	 \item \underline{\textit{F01}}: Primi approcci di ingegnerizzazione del prodotto sviluppato;
	 \item \underline{\textit{F02}}: Test del prodotto in un ambiente di produzione manifatturiero;
}