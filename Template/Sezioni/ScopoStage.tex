%----------------------------------------------------------------------------------------
%	STAGE DESCRIPTION
%----------------------------------------------------------------------------------------
\section*{Scopo dello stage}
% Personalizzare inserendo lo scopo dello stage, cioè una breve descrizione
Lo scopo di questo progetto di tesi è relativo allo studio e potenziale applicazione di tecnologie generative, in particolare approfondire le temaniche inerenti all'utilizzo di GANs (Generative Adversarial Network) o tecnologie equivalenti. L'applicazione va contestualizzata nel manifatturiero ed esempi tipici possono essere la generazione di nuove informazioni partendo da un set di base fornito. Ciò che si vuole valutare è l'applicabilità come: generazione di segnali sintetici da usare in campo AI/ML oppure generazione di nuove immagini.

Lo studente avrà il compito di ricercare e documentare tecniche e tecnologie generative per poi sviluppare un POC che consenta di mostrare le potenzialità in un contesto manifatturiero. Gli obiettivi iniziali possono essere riassunti come: generare nuove immagini da un set dato, aumentare la risoluzione di un'immagine o generazione di segnali digitali.

